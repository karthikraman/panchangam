% !Tex program = xelatex
\documentclass[12pt]{article}
\usepackage[dvipsnames]{xcolor} 
\usepackage[paperwidth=126mm,paperheight=168mm,left=2mm,right=2mm,top=8mm,bottom=1mm]{geometry}
\usepackage[xetex]{graphicx}
\usepackage{array}
\usepackage{setspace}
\usepackage{multirow}
\usepackage{pxfonts}
\usepackage{bbding}
\usepackage{wasysym} 
\usepackage{fontspec}
\usepackage{multicol}
\usepackage{supertabular}
\usepackage{../templates/listofitems}
\newcommand{\yearname}{2015}
%%%%%%%%%%%%%%%%%%%%%%%%%%%%%%%%%%%%%%%%%%%%%%%%%%%%%%%%%%%%%%%%%%%%%%%%%%%%%%%
%% MOONPHASE CODE 
%%%%%%%%%%%%%%%%%%%%%%%%%%%%%%%%%%%%%%%%%%%%%%%%%%%%%%%%%%%%%%%%%%%%%%%%%%%%%%%
%Credits: http://tex.stackexchange.com/questions/34785/how-to-typeset-moon-phase-symbols (Jake!)
\usepackage{tikz}
\usetikzlibrary{calendar,fpu}

\tikzset{
    moon colour/.style={
        moon fill/.style={
            fill=#1
        }
    },
    sky colour/.style={
        sky draw/.style={
            draw=#1
        },
        sky fill/.style={
            fill=#1
        }
    },
    southern hemisphere/.style={
        rotate=180
    }
}

\makeatletter
\pgfcalendardatetojulian{2010-01-15}{\c@pgf@counta} % 2010-01-15 07:11 UTC -- http://aa.usno.navy.mil/cgi-bin/aa_moonphases.pl?year=2010&ZZZ=END
\def\synodicmonth{29.530588853}
\newcommand{\moon}[2][]{%
    \edef\checkfordate{\noexpand\in@{-}{#2}}%
    \checkfordate%
    \ifin@%
        \pgfcalendardatetojulian{#2}{\c@pgf@countb}%
        \pgfkeys{/pgf/fpu=true,/pgf/fpu/output format=fixed}%
        \pgfmathsetmacro\dayssincenewmoon{\the\c@pgf@countb-\the\c@pgf@counta-(7/24+11/(24*60))}%
        \pgfmathsetmacro\lunarage{mod(\dayssincenewmoon,\synodicmonth)}
        \pgfkeys{/pgf/fpu=false}%%
    \else%
        \def\lunarage{#2}%
    \fi%
    \pgfmathsetmacro\leftside{ifthenelse(\lunarage<=\synodicmonth/2,cos(360*(\lunarage/\synodicmonth)),1)}%
    \pgfmathsetmacro\rightside{ifthenelse(\lunarage<=\synodicmonth/2,-1,-cos(360*(\lunarage/\synodicmonth))}%
    \tikz [moon colour=white,sky colour=black,#1]{
        \draw [moon fill, sky draw] (0,0) circle [radius=1ex];
        \draw [sky draw, sky fill] (0,1ex)
            arc (90:-90:\rightside ex and 1ex)
            arc (-90:90:\leftside ex and 1ex)
            -- cycle;
    }%
}
%%%%%%%%%%%%%%%%%%%%%%%%%%%%%%%%%%%%%%%%%%%%%%%%%%%%%%%%%%%%%%%%%%%%%%%%%%%%%%%
%% END MOONPHASE CODE
%%%%%%%%%%%%%%%%%%%%%%%%%%%%%%%%%%%%%%%%%%%%%%%%%%%%%%%%%%%%%%%%%%%%%%%%%%%%%%%
% \setlength{\footskip}{0mm}
% PDF SETUP
% ---- FILL IN HERE THE DOC TITLE AND AUTHOR
\defaultfontfeatures{Scale=MatchLowercase,Mapping=tex-text}
\setmainfont{siddhanta.ttf}[Path=../fonts/,Script=Devanagari] 
\setsansfont[Path=../fonts/,Scale=0.95,Numbers=Lining]{AlegreyaSans-Regular.ttf}
% \newfontfamily\noto[Path=../fonts/, Ligatures=TeX]{NotoSansUI-Regular}
%%%%%%% Numbers and counters %%%%%%%
\newcount\num
\newcount\tempone \newcount\temptwo
\newcommand{\devanumber}[1]{%
\num=#1\devanumberrecurse}

\newcommand{\devanumberrecurse}{%
{\tempone=\num
%  \showthe\tempone\ %
\ifnum\num > 0 
    \divide \num by 10%
    \temptwo=\num \multiply\temptwo by -10%
    \devanumberrecurse%
%   \\stage 2\ %
%   \showthe\temptwo\ %
%   temp1=\number\tempone\ %
%   num=\number\num\ %
    \advance\tempone by \temptwo%
    \devadigit
\fi
}}
\newcommand{\devadigit}{%
\ifcase\tempone०\or१\or२\or३\or४\or५\or६\or७\or८\or९\fi%\number\tempone%
}
\newcommand{\eventsep}{~\raisebox{1pt}{\scriptsize$\Diamondblack$} }
\newcommand{\TO}{\hspace{1pt}\raisebox{1pt}{\footnotesize\RIGHTarrow}\hspace{1pt}}
\newcommand{\To}{\hspace{1pt}\raisebox{1pt}{\footnotesize\RIGHTarrow}\hspace{1pt}}
\newcommand{\Too}{\hspace{1pt}\raisebox{1pt}{\footnotesize\RIGHTarrow\hspace{-5pt}\RIGHTarrow}\hspace{1pt}}
%%%%%%%% Calendar display stuff %%%%%%%%%%%
\newcommand{\samvatsaraName}{}
\newcommand{\solarMonthName}{}
\newcommand{\solarMonthEndTime}{}
\newcommand{\lunarMonthName}{}
\newcommand{\solarMonthDate}{}
\newcommand{\vaaraName}{}
\newcommand{\rtuName}{}
\newcommand{\ayanamName}{}

\newcommand{\sunmonth}[8]{%
\renewcommand{\samvatsaraName}{#6}
\renewcommand{\solarMonthName}{#1}
\renewcommand{\solarMonthEndTime}{#3}
\renewcommand{\lunarMonthName}{#4}
\renewcommand{\solarMonthDate}{#2}
\renewcommand{\vaaraName}{#5}
\renewcommand{\ayanamName}{#7}
\renewcommand{\rtuName}{#8}
}
\newcommand{\tamil}[1]{%
{\fontspec[Scale=0.8,FakeStretch=0.9,Path=../fonts/]{NotoSansTamil-Regular.ttf} \footnotesize #1}}
\newcommand{\kalas}[1]{%
\setsepchar{ }
\readlist\arg{#1}
{\small{\mbox{ब्राह्म\,\textsf{\arg[1]}{\scriptsize\RIGHTarrow}\,सङ्गव\,\textsf{\arg[4]}{\scriptsize\RIGHTarrow}\,मध्याह्न\,\textsf{\arg[7]}{\scriptsize\RIGHTarrow}\,अपराह्ण\,\textsf{\arg[8]}{\scriptsize\RIGHTarrow}\,सायाह्न\,\textsf{\arg[9]}{\scriptsize\RIGHTarrow}}\hfill {दिनान्तः{\scriptsize\RIGHTarrow}\textsf{\arg[14]}}}}\\[-1ex]
{\small{\mbox{प्रातः सन्ध्या \textsf{\arg[2]}{\scriptsize\RIGHTarrow}\textsf{\arg[3]} माध्याह्निक \textsf{\arg[5]}{\scriptsize\RIGHTarrow}\textsf{\arg[6]} 
सायं \textsf{\arg[10]}{\scriptsize\RIGHTarrow}\textsf{\arg[11]}}\hfill शयन \textsf{\arg[12]}{\scriptsize\RIGHTarrow}\textsf{\arg[13]}}\\[-4.5ex]}
}
\newcommand{\sunmoonrsdata}[5]{%
\mbox{\large\sun{\small\UParrow}\textsf{#1}~\sun{\small\DOWNarrow}\textsf{#2}}\hfill
\mbox{\large\rightmoon{\small\UParrow}\textsf{#3}~\rightmoon{\small\DOWNarrow}\textsf{#4}}\\
#5
 }
\newcommand{\sunmoonsrdata}[5]{%
\mbox{\large\sun{\small\UParrow}\textsf{#1}~\sun{\small\DOWNarrow}\textsf{#2}}\hfill
\mbox{\large\rightmoon{\small\DOWNarrow}\textsf{#4}~\rightmoon{\small\UParrow}\textsf{#3}}\\
#5
 }
\newcommand{\tnykdata}[6]{\large
{#1}\\[-3pt]%Tithi
{नक्षत्रम्–#2\hspace{2ex}\parbox[c][2em][c]{10mm}{\scriptsize(#3)}}\\[-3pt]%Nakshatram and Rashi
{\setstretch{0.55}
\begin{tabular}{@{}r@{}p{108mm}@{}}
योगः–&#4\\[2pt]%Yogam
करणम्–&#5\\%Karanam
\end{tabular}}
\parbox[c][2ex][c]{0.9\linewidth}{#6}
}
\newcommand{\rygdata}[3]{%
\begin{minipage}{\linewidth}
\centering
\rule[-1ex]{0.7\textwidth}{.4pt}
\small राहु॰~\textsf{#1}~~यम॰~\textsf{#2}~~गुलिक॰~\textsf{#3}%Rahu Yama Gulika
\end{minipage}
}
\newcommand{\caldata}[8]{%
#3% Calls \sunmonth
\large% Fixes font size
{\centering\begin{tabular}{c|c}
\large \textsf{\yearname} & {\large\samvatsaraName}\\[-1ex]%YYYY
& {\footnotesize \ayanamName \hspace{6pt} \rtuName}\\[0.4ex]
\mbox{\sffamily\fontsize{20}{25}\selectfont {\uppercase{#1}}} & \parbox[2.6cm]{1cm}{\LARGE\solarMonthName}\\%mmm
& \mbox{\small \solarMonthEndTime}\\
& \parbox{3cm}{\centering\lunarMonthName}\\
\hspace{0.465\linewidth} & \hspace{0.465\linewidth} \\
\mbox{\sffamily\fontsize{96}{115}\selectfont #2} & \mbox{\fontsize{90}{24}\selectfont \devanumber{\solarMonthDate}}\\[2.2ex]%DD
\mbox{\sffamily\fontsize{24}{28}\selectfont\uppercase{#8}} & \parbox[2.6cm]{1cm}{\LARGE\vaaraName}\\[1.2ex]%Day of the week
\hline
\end{tabular}
}\mbox{}\\[1pt]
#4\\[0.5em]%Sun rise, kalas etc
#5\\[1.6em]%Tithi, Nakshatram, Varam, Yogam
\parbox[c][4em][c]{0.95\linewidth}{\centering\normalsize\textcolor{RoyalBlue}{#7}}\\\vfill%Festivals
#6%Rahu, Yama etc
\clearpage
}

\addtolength{\headsep}{-3ex}
\setlength\parindent{0pt}
\pagestyle{empty}
\begin{document}
\mbox{}
\renewcommand{\yearname}{2020}
\begin{center}
{\sffamily \fontsize{80}{80}\selectfont  2020\\[0.5cm]}
\mbox{\fontsize{48}{48}\selectfont विकारी–शार्वरी}\\
\mbox{\fontsize{32}{32}\selectfont कलि } %
{\sffamily \fontsize{43}{43}\selectfont  5120–5121\\[0.5cm]}
\hrule
\vspace{0.2cm}
{\sffamily \fontsize{50}{50}\selectfont  \uppercase{Chennai}\\[0.2cm]}
{\sffamily \fontsize{23}{23}\selectfont  {13.090000°N, 80.270000°E}\\[0.2cm]}
\hrule
\end{center}
\clearpage
\caldata{JANUARY}{1}{\sunmonth{धनुः}{17}{}{पौषः\\[-8pt]{\scriptsize (हेमन्त-ऋतुः)}}{सौम्यः}{विकारी}{दक्षिणायनम्}{हेमन्त-ऋतुः}}
{\sunmoonrsdata{06:34}{17:49}{11:06}{23:04}
{\kalas{04:52 05:43 09:34 08:49 10:19 16:19 11:04 13:19 15:34 17:04 18:40 21:00 22:36 01:48(+1)}}}
{\tnykdata{\mbox{\raisebox{-1pt}{\moon[scale=0.8]{6}}\hspace{2pt}शुक्ल-षष्ठी\To{}\textsf{29-42 (18:27)}}\hspace{1ex}}%
{\mbox{पूर्वप्रोष्ठपदा\To{}\textsf{54-31 (04:23(+1))}}}{\mbox{कुम्भ-राशिः \RIGHTarrow \textsf{21:38}}}%
{\mbox{व्यतीपातः\To{}\textsf{38-10 (21:50)}}\hspace{1ex}\mbox{वरीयान्\Too{}}}%
{\mbox{तैतिलः\To{}\textsf{29-42 (18:27)}}\hspace{1ex}\mbox{गरः\Too{}}}{\scriptsize }
}
{\rygdata{12:12--13:36}{07:59--09:23}{10:47--12:12}}
{षष्ठी-व्रतम्\eventsep महाधनुर्व्यतीपातम्\eventsep व्यतीपात-श्राद्धम्}
{Wed} 
\caldata{JANUARY}{2}{\sunmonth{धनुः}{18}{}{पौषः\\[-8pt]{\scriptsize (हेमन्त-ऋतुः)}}{गुरुः}{विकारी}{दक्षिणायनम्}{हेमन्त-ऋतुः}}
{\sunmoonrsdata{06:35}{17:50}{11:43}{23:50}
{\kalas{04:53 05:44 09:35 08:50 10:20 16:20 11:05 13:20 15:35 17:05 18:41 21:01 22:37 01:48(+1)}}}
{\tnykdata{\mbox{\raisebox{-1pt}{\moon[scale=0.8]{7}}\hspace{2pt}शुक्ल-सप्तमी\To{}\textsf{36-3 (21:00)}}\hspace{1ex}}%
{\mbox{उत्तरप्रोष्ठपदा\To{}अहोरात्रम्}}{\mbox{मीन-राशिः}}%
{\mbox{वरीयान्\To{}\textsf{40-14 (22:40)}}\hspace{1ex}\mbox{परिघः\Too{}}}%
{\mbox{गरः\To{}\textsf{2-52 (07:44)}}\hspace{1ex}\mbox{वणिजः\To{}\textsf{36-3 (21:00)}}\hspace{1ex}\mbox{विष्टिः\Too{}}}{\scriptsize }
}
{\rygdata{13:37--15:01}{06:35--07:59}{09:23--10:48}}
{\tamil{உந்து~மதக்களிற்றன்{}}}
{Thu} 
\caldata{JANUARY}{3}{\sunmonth{धनुः}{19}{}{पौषः\\[-8pt]{\scriptsize (हेमन्त-ऋतुः)}}{भृगुः}{विकारी}{दक्षिणायनम्}{हेमन्त-ऋतुः}}
{\sunmoonrsdata{06:35}{17:50}{12:19}{00:35(+1)}
{\kalas{04:53 05:44 09:35 08:50 10:20 16:20 11:05 13:20 15:35 17:05 18:41 21:01 22:37 01:48(+1)}}}
{\tnykdata{\mbox{\raisebox{-1pt}{\moon[scale=0.8]{8}}\hspace{2pt}शुक्ल-अष्टमी\To{}\textsf{42-8 (23:26)}}\hspace{1ex}}%
{\mbox{उत्तरप्रोष्ठपदा\To{}\textsf{1-52 (07:20)}}}{\mbox{मीन-राशिः}}%
{\mbox{परिघः\To{}\textsf{42-4 (23:25)}}\hspace{1ex}\mbox{शिवः\Too{}}}%
{\mbox{विष्टिः\To{}\textsf{9-9 (10:15)}}\hspace{1ex}\mbox{बवः\To{}\textsf{42-8 (23:26)}}\hspace{1ex}\mbox{बालवः\Too{}}}{\scriptsize }
}
{\rygdata{10:48--12:13}{15:01--16:26}{07:59--09:24}}
{भृगुरेवती-पुण्यकालः}
{Fri} 
\caldata{JANUARY}{4}{\sunmonth{धनुः}{20}{}{पौषः\\[-8pt]{\scriptsize (हेमन्त-ऋतुः)}}{स्थिरः}{विकारी}{दक्षिणायनम्}{हेमन्त-ऋतुः}}
{\sunmoonrsdata{06:35}{17:51}{12:56}{01:22(+1)}
{\kalas{04:53 05:44 09:35 08:50 10:21 16:21 11:06 13:21 15:36 17:06 18:42 21:02 22:38 01:49(+1)}}}
{\tnykdata{\mbox{\raisebox{-1pt}{\moon[scale=0.8]{9}}\hspace{2pt}शुक्ल-नवमी\To{}\textsf{47-22 (01:32(+1))}}\hspace{1ex}}%
{\mbox{रेवती\To{}\textsf{8-45 (10:05)}}}{\mbox{मीन-राशिः \RIGHTarrow \textsf{10:05}}}%
{\mbox{शिवः\To{}\textsf{43-16 (23:54)}}\hspace{1ex}\mbox{सिद्धिः\Too{}}}%
{\mbox{बालवः\To{}\textsf{14-52 (12:33)}}\hspace{1ex}\mbox{कौलवः\To{}\textsf{47-22 (01:32(+1))}}\hspace{1ex}\mbox{तैतिलः\Too{}}}{\scriptsize }
}
{\rygdata{09:24--10:49}{13:37--15:02}{06:35--08:00}}
{\tamil{வாயிலார் நாயனார் (49) குருபூஜை}}
{Sat} 
\caldata{JANUARY}{5}{\sunmonth{धनुः}{21}{}{पौषः\\[-8pt]{\scriptsize (हेमन्त-ऋतुः)}}{भानुः}{विकारी}{दक्षिणायनम्}{हेमन्त-ऋतुः}}
{\sunmoonrsdata{06:36}{17:51}{13:34}{02:10(+1)}
{\kalas{04:54 05:45 09:36 08:51 10:21 16:21 11:06 13:21 15:36 17:06 18:42 21:03 22:38 01:49(+1)}}}
{\tnykdata{\mbox{\raisebox{-1pt}{\moon[scale=0.8]{10}}\hspace{2pt}शुक्ल-दशमी\To{}\textsf{51-17 (03:07(+1))}}\hspace{1ex}}%
{\mbox{अश्विनी\To{}\textsf{14-38 (12:27)}}}{\mbox{मेष-राशिः}}%
{\mbox{सिद्धिः\To{}\textsf{43-31 (00:00(+1))}}\hspace{1ex}\mbox{साध्यः\Too{}}}%
{\mbox{तैतिलः\To{}\textsf{19-30 (14:24)}}\hspace{1ex}\mbox{गरः\To{}\textsf{51-17 (03:07(+1))}}\hspace{1ex}\mbox{वणिजः\Too{}}}{\scriptsize }
}
{\rygdata{16:27--17:51}{12:14--13:38}{15:02--16:27}}
{}
{Sun} 
\caldata{JANUARY}{6}{\sunmonth{धनुः}{22}{}{पौषः\\[-8pt]{\scriptsize (हेमन्त-ऋतुः)}}{इन्दुः}{विकारी}{दक्षिणायनम्}{हेमन्त-ऋतुः}}
{\sunmoonrsdata{06:36}{17:52}{14:15}{03:00(+1)}
{\kalas{04:54 05:45 09:36 08:51 10:21 16:22 11:06 13:22 15:37 17:07 18:43 21:03 22:39 01:50(+1)}}}
{\tnykdata{\mbox{\raisebox{-1pt}{\moon[scale=0.8]{11}}\hspace{2pt}शुक्ल-एकादशी\To{}\textsf{53-34 (04:02(+1))}}\hspace{1ex}}%
{\mbox{अपभरणी\To{}\textsf{19-8 (14:15)}}}{\mbox{मेष-राशिः \RIGHTarrow \textsf{20:36}}}%
{\mbox{साध्यः\To{}\textsf{42-34 (23:38)}}\hspace{1ex}\mbox{शुभः\Too{}}}%
{\mbox{वणिजः\To{}\textsf{22-38 (15:40)}}\hspace{1ex}\mbox{विष्टिः\To{}\textsf{53-34 (04:02(+1))}}\hspace{1ex}\mbox{बवः\Too{}}}{\scriptsize }
}
{\rygdata{08:01--09:25}{10:49--12:14}{13:38--15:03}}
{काञ्ची ५५ जगद्गुरु श्री~चन्द्रचूडेन्द्र सरस्वती ३ आराधना~\#{४९६}\eventsep मन्वादि~(चाक्षुषः~[६])\eventsep सर्व-पुत्रदा-एकादशी\eventsep सर्व-वैकुण्ठ-एकादशी\eventsep त्रैलङ्ग-स्वामी~जयन्ती}
{Mon} 
\caldata{JANUARY}{7}{\sunmonth{धनुः}{23}{}{पौषः\\[-8pt]{\scriptsize (हेमन्त-ऋतुः)}}{भौमः}{विकारी}{दक्षिणायनम्}{हेमन्त-ऋतुः}}
{\sunmoonrsdata{06:36}{17:52}{15:00}{03:54(+1)}
{\kalas{04:54 05:45 09:37 08:52 10:22 16:22 11:07 13:22 15:37 17:07 18:43 21:04 22:39 01:50(+1)}}}
{\tnykdata{\mbox{\raisebox{-1pt}{\moon[scale=0.8]{12}}\hspace{2pt}शुक्ल-द्वादशी\To{}\textsf{54-4 (04:14(+1))}}\hspace{1ex}}%
{\mbox{कृत्तिका\To{}\textsf{21-59 (15:24)}}}{\mbox{वृषभ-राशिः}}%
{\mbox{शुभः\To{}\textsf{40-15 (22:43)}}\hspace{1ex}\mbox{शुक्लः\Too{}}}%
{\mbox{बवः\To{}\textsf{24-2 (16:14)}}\hspace{1ex}\mbox{बालवः\To{}\textsf{54-4 (04:14(+1))}}\hspace{1ex}\mbox{कौलवः\Too{}}}{\scriptsize }
}
{\rygdata{15:03--16:28}{09:25--10:50}{12:14--13:39}}
{हरिवासरः\textsf{}{\RIGHTarrow}\textsf{10:09}\eventsep कृत्तिका-व्रतम्}
{Tue} 
\caldata{JANUARY}{8}{\sunmonth{धनुः}{24}{}{पौषः\\[-8pt]{\scriptsize (हेमन्त-ऋतुः)}}{सौम्यः}{विकारी}{दक्षिणायनम्}{हेमन्त-ऋतुः}}
{\sunmoonrsdata{06:37}{17:53}{15:49}{04:51(+1)}
{\kalas{04:55 05:46 09:37 08:52 10:22 16:23 11:07 13:22 15:38 17:08 18:44 21:04 22:40 01:50(+1)}}}
{\tnykdata{\mbox{\raisebox{-1pt}{\moon[scale=0.8]{13}}\hspace{2pt}शुक्ल-त्रयोदशी\To{}\textsf{52-48 (03:44(+1))}}\hspace{1ex}}%
{\mbox{रोहिणी\To{}\textsf{23-5 (15:51)}}}{\mbox{वृषभ-राशिः \RIGHTarrow \textsf{03:49(+1)}}}%
{\mbox{शुक्लः\To{}\textsf{36-34 (21:14)}}\hspace{1ex}\mbox{ब्रह्म\Too{}}}%
{\mbox{कौलवः\To{}\textsf{23-39 (16:04)}}\hspace{1ex}\mbox{तैतिलः\To{}\textsf{52-48 (03:44(+1))}}\hspace{1ex}\mbox{गरः\Too{}}}{\scriptsize }
}
{\rygdata{12:15--13:39}{08:01--09:26}{10:50--12:15}}
{प्रदोष-व्रतम्}
{Wed} 
\caldata{JANUARY}{9}{\sunmonth{धनुः}{25}{}{पौषः\\[-8pt]{\scriptsize (हेमन्त-ऋतुः)}}{गुरुः}{विकारी}{दक्षिणायनम्}{हेमन्त-ऋतुः}}
{\sunmoonrsdata{06:37}{17:54}{16:43}{05:50(+1)}
{\kalas{04:55 05:46 09:37 08:52 10:22 16:23 11:08 13:23 15:38 17:09 18:45 21:05 22:40 01:51(+1)}}}
{\tnykdata{\mbox{\raisebox{-1pt}{\moon[scale=0.8]{14}}\hspace{2pt}शुक्ल-चतुर्दशी\To{}\textsf{49-53 (02:34(+1))}}\hspace{1ex}}%
{\mbox{मृगशीर्षम्\To{}\textsf{22-31 (15:38)}}}{\mbox{मिथुन-राशिः}}%
{\mbox{ब्रह्म\To{}\textsf{31-33 (19:14)}}\hspace{1ex}\mbox{इन्द्रः\Too{}}}%
{\mbox{गरः\To{}\textsf{21-31 (15:14)}}\hspace{1ex}\mbox{वणिजः\To{}\textsf{49-53 (02:34(+1))}}\hspace{1ex}\mbox{विष्टिः\Too{}}}{\scriptsize }
}
{\rygdata{13:40--15:04}{06:37--08:02}{09:26--10:51}}
{अन्धकासुर-वधः\eventsep काञ्ची ८ जगद्गुरु श्री~कैवल्यानन्दयोगेन्द्र सरस्वती आराधना~\#{१९९१}}
{Thu} 
\caldata{JANUARY}{10}{\sunmonth{धनुः}{26}{}{पौषः\\[-8pt]{\scriptsize (हेमन्त-ऋतुः)}}{भृगुः}{विकारी}{दक्षिणायनम्}{हेमन्त-ऋतुः}}
{\sunmoonrsdata{06:37}{17:54}{17:41}{---}
{\kalas{04:55 05:46 09:38 08:53 10:23 16:24 11:08 13:23 15:39 17:09 18:45 21:05 22:40 01:51(+1)}}}
{\tnykdata{\mbox{\raisebox{-1pt}{\moon[scale=0.8]{15}}\hspace{2pt}पौर्णमासी\To{}\textsf{45-34 (00:51(+1))}}\hspace{1ex}}%
{\mbox{आर्द्रा\To{}\textsf{20-28 (14:49)}}}{\mbox{मिथुन-राशिः}}%
{\mbox{इन्द्रः\To{}\textsf{25-22 (16:46)}}\hspace{1ex}\mbox{वैधृतिः\Too{}}}%
{\mbox{विष्टिः\To{}\textsf{17-52 (13:46)}}\hspace{1ex}\mbox{बवः\To{}\textsf{45-34 (00:51(+1))}}\hspace{1ex}\mbox{बालवः\Too{}}}{\scriptsize }
}
{\rygdata{10:51--12:16}{15:05--16:30}{08:02--09:26}}
{बदरी ज्योतिर्मठ-प्रतिष्ठापन~जयन्ती~\#{२५०५}\eventsep पूर्णिमा~व्रतम्\eventsep \tamil{ஆருத்ரா~தரிசனம்/நடராஜர் மஹாபிஷேகம்}\eventsep \tamil{சடைய நாயனார் (61) குருபூஜை}\eventsep वेङ्कटाचले पूर्णिमा~गरुड-सेवा\eventsep शाकम्भरी~जयन्ती\eventsep शृङ्गेरी शारदामठ-प्रतिष्ठापन~जयन्ती~\#{२५०३}}
{Fri} 
\caldata{JANUARY}{11}{\sunmonth{धनुः}{27}{}{पौषः\\[-8pt]{\scriptsize (हेमन्त-ऋतुः)}}{स्थिरः}{विकारी}{दक्षिणायनम्}{हेमन्त-ऋतुः}}
{\sunmoonsrdata{06:37}{17:55}{18:42}{06:49}
{\kalas{04:56 05:47 09:38 08:53 10:23 16:24 11:08 13:24 15:39 17:10 18:46 21:05 22:41 01:52(+1)}}}
{\tnykdata{\mbox{\raisebox{-1pt}{\moon[scale=0.8]{16}}\hspace{2pt}कृष्ण-प्रथमा\To{}\textsf{40-8 (22:41)}}\hspace{1ex}}%
{\mbox{पुनर्वसुः\To{}\textsf{17-11 (13:30)}}}{\mbox{मिथुन-राशिः \RIGHTarrow \textsf{07:52}}}%
{\mbox{वैधृतिः\To{}\textsf{18-13 (13:55)}}\hspace{1ex}\mbox{विष्कम्भः\Too{}}}%
{\mbox{बालवः\To{}\textsf{12-58 (11:49)}}\hspace{1ex}\mbox{कौलवः\To{}\textsf{40-8 (22:41)}}\hspace{1ex}\mbox{तैतिलः\Too{}}}{\scriptsize }
}
{\rygdata{09:27--10:51}{13:41--15:05}{06:37--08:02}}
{रमण~महर्षि~जयन्ती~\#{१४१}\eventsep \tamil{கூடாரவல்லீ}\eventsep वैधृति-श्राद्धम्}
{Sat} 
\caldata{JANUARY}{12}{\sunmonth{धनुः}{28}{}{पौषः\\[-8pt]{\scriptsize (हेमन्त-ऋतुः)}}{भानुः}{विकारी}{दक्षिणायनम्}{हेमन्त-ऋतुः}}
{\sunmoonsrdata{06:38}{17:55}{19:44}{07:46}
{\kalas{04:56 05:47 09:38 08:53 10:24 16:25 11:09 13:24 15:40 17:10 18:46 21:06 22:41 01:52(+1)}}}
{\tnykdata{\mbox{\raisebox{-1pt}{\moon[scale=0.8]{17}}\hspace{2pt}कृष्ण-द्वितीया\To{}\textsf{33-55 (20:12)}}\hspace{1ex}}%
{\mbox{पुष्यः\To{}\textsf{12-59 (11:50)}}}{\mbox{कटक-राशिः}}%
{\mbox{विष्कम्भः\To{}\textsf{10-19 (10:46)}}\hspace{1ex}\mbox{प्रीतिः\Too{}}}%
{\mbox{तैतिलः\To{}\textsf{7-6 (09:28)}}\hspace{1ex}\mbox{गरः\To{}\textsf{33-55 (20:12)}}\hspace{1ex}\mbox{वणिजः\Too{}}}{\scriptsize }
}
{\rygdata{16:31--17:55}{12:16--13:41}{15:06--16:31}}
{काञ्ची ३७ जगद्गुरु श्री~विद्याघनेन्द्र सरस्वती ३ आराधना~\#{१२३२}\eventsep काञ्ची ६२ जगद्गुरु श्री~चन्द्रशेखरेन्द्र सरस्वती ४ आराधना~\#{२३७}\eventsep रविपुष्ययोग-पुण्यकालः\eventsep \tamil{கறவைகள் பின்சென்று}}
{Sun} 
\caldata{JANUARY}{13}{\sunmonth{धनुः}{29}{}{पौषः\\[-8pt]{\scriptsize (हेमन्त-ऋतुः)}}{इन्दुः}{विकारी}{दक्षिणायनम्}{हेमन्त-ऋतुः}}
{\sunmoonsrdata{06:38}{17:56}{20:45}{08:40}
{\kalas{04:56 05:47 09:39 08:53 10:24 16:25 11:09 13:25 15:40 17:11 18:47 21:06 22:42 01:52(+1)}}}
{\tnykdata{\mbox{\raisebox{-1pt}{\moon[scale=0.8]{18}}\hspace{2pt}कृष्ण-तृतीया\To{}\textsf{27-15 (17:32)}}\hspace{1ex}}%
{\mbox{आश्रेषा\To{}\textsf{8-13 (09:55)}}}{\mbox{कटक-राशिः \RIGHTarrow \textsf{09:55}}}%
{\mbox{प्रीतिः\To{}\textsf{1-57 (07:25)}}\hspace{1ex}\mbox{आयुष्मान्\To{}\textsf{53-21 (03:59(+1))}}\hspace{1ex}\mbox{सौभाग्यः\Too{}}}%
{\mbox{वणिजः\To{}\textsf{0-37 (06:53)}}\hspace{1ex}\mbox{विष्टिः\To{}\textsf{27-15 (17:32)}}\hspace{1ex}\mbox{बवः\To{}\textsf{53-51 (04:11(+1))}}\hspace{1ex}\mbox{बालवः\Too{}}}{\scriptsize }
}
{\rygdata{08:03--09:27}{10:52--12:17}{13:42--15:06}}
{लम्बोदर-महागणपति सङ्कटहर-चतुर्थी-व्रतम्}
{Mon} 
\caldata{JANUARY}{14}{\sunmonth{धनुः}{30}{\mbox{धनुः{\tiny\RIGHTarrow}\textsf{02:14(+1)}}}{पौषः\\[-8pt]{\scriptsize (हेमन्त-ऋतुः)}}{भौमः}{विकारी}{दक्षिणायनम्}{हेमन्त-ऋतुः}}
{\sunmoonsrdata{06:38}{17:56}{21:44}{09:31}
{\kalas{04:56 05:47 09:39 08:54 10:24 16:26 11:09 13:25 15:41 17:11 18:47 21:07 22:42 01:53(+1)}}}
{\tnykdata{\mbox{\raisebox{-1pt}{\moon[scale=0.8]{19}}\hspace{2pt}कृष्ण-चतुर्थी\To{}\textsf{20-27 (14:49)}}\hspace{1ex}}%
{\mbox{मघा\To{}\textsf{3-12 (07:55)}}\hspace{1ex}\mbox{पूर्वफल्गुनी\To{}\textsf{58-16 (05:57(+1))}}}{\mbox{सिंह-राशिः}}%
{\mbox{सौभाग्यः\To{}\textsf{44-46 (00:33(+1))}}\hspace{1ex}\mbox{शोभनः\Too{}}}%
{\mbox{बालवः\To{}\textsf{20-27 (14:49)}}\hspace{1ex}\mbox{कौलवः\To{}\textsf{47-6 (01:29(+1))}}\hspace{1ex}\mbox{तैतिलः\Too{}}}{\scriptsize }
}
{\rygdata{15:07--16:32}{09:28--10:52}{12:17--13:42}}
{अङ्गारक-चतुर्थी\eventsep \tamil{போகி}}
{Tue} 
\caldata{JANUARY}{15}{\sunmonth{मकरः}{1}{}{पौषः\\[-8pt]{\scriptsize (हेमन्त-ऋतुः)}}{सौम्यः}{विकारी}{उत्तरायणम्}{हेमन्त-ऋतुः}}
{\sunmoonsrdata{06:38}{17:57}{22:41}{10:19}
{\kalas{04:57 05:47 09:39 08:54 10:24 16:26 11:10 13:25 15:41 17:12 18:48 21:07 22:43 01:53(+1)}}}
{\tnykdata{\mbox{\raisebox{-1pt}{\moon[scale=0.8]{20}}\hspace{2pt}कृष्ण-पञ्चमी\To{}\textsf{13-49 (12:10)}}\hspace{1ex}}%
{\mbox{उत्तरफल्गुनी\To{}\textsf{53-40 (04:07(+1))}}}{\mbox{सिंह-राशिः \RIGHTarrow \textsf{11:28}}}%
{\mbox{शोभनः\To{}\textsf{36-25 (21:13)}}\hspace{1ex}\mbox{अतिगण्डः\Too{}}}%
{\mbox{तैतिलः\To{}\textsf{13-49 (12:10)}}\hspace{1ex}\mbox{गरः\To{}\textsf{40-39 (22:54)}}\hspace{1ex}\mbox{वणिजः\Too{}}}{\scriptsize }
}
{\rygdata{12:18--13:42}{08:03--09:28}{10:53--12:18}}
{बहुल-पञ्चमी\eventsep मकर-ज्योतिः\eventsep मकर-सङ्क्रान्तिः/उत्तरायण-पुण्यकालः\eventsep \tamil{சண்டேஶ்வர நாயனார் (19) குருபூஜை}\eventsep \tamil{மதுரை மீனாக்ஷீ கோயிலில் கல் யானைக்கு கரும்பு கோடுத்த லீலை}\eventsep त्यागराज-आराधना~\#{१७३}}
{Wed} 
\caldata{JANUARY}{16}{\sunmonth{मकरः}{2}{}{पौषः\\[-8pt]{\scriptsize (हेमन्त-ऋतुः)}}{गुरुः}{विकारी}{उत्तरायणम्}{हेमन्त-ऋतुः}}
{\sunmoonsrdata{06:38}{17:58}{23:37}{11:04}
{\kalas{04:57 05:48 09:40 08:54 10:25 16:27 11:10 13:26 15:42 17:12 18:48 21:08 22:43 01:53(+1)}}}
{\tnykdata{\mbox{\raisebox{-1pt}{\moon[scale=0.8]{21}}\hspace{2pt}कृष्ण-षष्ठी\To{}\textsf{7-37 (09:41)}}\hspace{1ex}}%
{\mbox{हस्तः\To{}\textsf{49-40 (02:31(+1))}}}{\mbox{कन्या-राशिः}}%
{\mbox{अतिगण्डः\To{}\textsf{28-29 (18:02)}}\hspace{1ex}\mbox{सुकर्म\Too{}}}%
{\mbox{वणिजः\To{}\textsf{7-37 (09:41)}}\hspace{1ex}\mbox{विष्टिः\To{}\textsf{34-44 (20:32)}}\hspace{1ex}\mbox{बवः\Too{}}}{\scriptsize }
}
{\rygdata{13:43--15:08}{06:38--08:03}{09:28--10:53}}
{इन्द्र-पूजा/गो-पूजा\eventsep \tamil{கனுப்~பொங்கல்}\eventsep श्री~शेषाद्रि-स्वामी~जयन्ती~\#{१५१}}
{Thu} 
\caldata{JANUARY}{17}{\sunmonth{मकरः}{3}{}{पौषः\\[-8pt]{\scriptsize (हेमन्त-ऋतुः)}}{भृगुः}{विकारी}{उत्तरायणम्}{हेमन्त-ऋतुः}}
{\sunmoonsrdata{06:39}{17:58}{00:32(+1)}{11:48}
{\kalas{04:57 05:48 09:40 08:54 10:25 16:27 11:10 13:26 15:42 17:13 18:49 21:08 22:43 01:53(+1)}}}
{\tnykdata{\mbox{\raisebox{-1pt}{\moon[scale=0.8]{22}}\hspace{2pt}कृष्ण-सप्तमी\To{}\textsf{2-2 (07:28)}}\hspace{1ex}\mbox{\raisebox{-1pt}{\moon[scale=0.8]{23}}\hspace{2pt}कृष्ण-अष्टमी\To{}\textsf{57-16 (05:33(+1))}}\hspace{1ex}}%
{\mbox{चित्रा\To{}\textsf{46-24 (01:12(+1))}}}{\mbox{कन्या-राशिः \RIGHTarrow \textsf{13:49}}}%
{\mbox{सुकर्म\To{}\textsf{21-7 (15:06)}}\hspace{1ex}\mbox{धृतिः\Too{}}}%
{\mbox{बवः\To{}\textsf{2-2 (07:28)}}\hspace{1ex}\mbox{बालवः\To{}\textsf{29-33 (18:28)}}\hspace{1ex}\mbox{कौलवः\To{}\textsf{57-16 (05:33(+1))}}\hspace{1ex}\mbox{तैतिलः\Too{}}}{\scriptsize }
}
{\rygdata{10:53--12:18}{15:08--16:33}{08:03--09:28}}
{\tamil{தை~வெள்ளிக்கிழமை}\eventsep विवेकानन्द~जयन्ती~\#{१५८}}
{Fri} 
\caldata{JANUARY}{18}{\sunmonth{मकरः}{4}{}{पौषः\\[-8pt]{\scriptsize (हेमन्त-ऋतुः)}}{स्थिरः}{विकारी}{उत्तरायणम्}{हेमन्त-ऋतुः}}
{\sunmoonsrdata{06:39}{17:59}{01:28(+1)}{12:33}
{\kalas{04:57 05:48 09:40 08:55 10:25 16:28 11:11 13:27 15:43 17:13 18:49 21:09 22:44 01:54(+1)}}}
{\tnykdata{\mbox{\raisebox{-1pt}{\moon[scale=0.8]{24}}\hspace{2pt}कृष्ण-नवमी\To{}\textsf{53-24 (04:00(+1))}}\hspace{1ex}}%
{\mbox{स्वाती\To{}\textsf{44-1 (00:15(+1))}}}{\mbox{तुला-राशिः}}%
{\mbox{धृतिः\To{}\textsf{14-26 (12:25)}}\hspace{1ex}\mbox{शूलः\Too{}}}%
{\mbox{तैतिलः\To{}\textsf{25-13 (16:44)}}\hspace{1ex}\mbox{गरः\To{}\textsf{53-24 (04:00(+1))}}\hspace{1ex}\mbox{वणिजः\Too{}}}{\scriptsize }
}
{\rygdata{09:29--10:54}{13:44--15:09}{06:39--08:04}}
{भीष्म~जयन्ती}
{Sat} 
\caldata{JANUARY}{19}{\sunmonth{मकरः}{5}{}{पौषः\\[-8pt]{\scriptsize (हेमन्त-ऋतुः)}}{भानुः}{विकारी}{उत्तरायणम्}{हेमन्त-ऋतुः}}
{\sunmoonsrdata{06:39}{17:59}{02:24(+1)}{13:19}
{\kalas{04:57 05:48 09:40 08:55 10:26 16:28 11:11 13:27 15:43 17:14 18:50 21:09 22:44 01:54(+1)}}}
{\tnykdata{\mbox{\raisebox{-1pt}{\moon[scale=0.8]{25}}\hspace{2pt}कृष्ण-दशमी\To{}\textsf{50-30 (02:51(+1))}}\hspace{1ex}}%
{\mbox{विशाखा\To{}\textsf{42-35 (23:41)}}}{\mbox{तुला-राशिः \RIGHTarrow \textsf{17:48}}}%
{\mbox{शूलः\To{}\textsf{8-29 (10:02)}}\hspace{1ex}\mbox{गण्डः\Too{}}}%
{\mbox{वणिजः\To{}\textsf{21-49 (15:23)}}\hspace{1ex}\mbox{विष्टिः\To{}\textsf{50-30 (02:51(+1))}}\hspace{1ex}\mbox{बवः\Too{}}}{\scriptsize }
}
{\rygdata{16:34--17:59}{12:19--13:44}{15:09--16:34}}
{\tamil{திருநீலகண்ட நாயனார் (1) குருபூஜை}\eventsep त्रैलोक्य-गौरी-व्रतम्}
{Sun} 
\caldata{JANUARY}{20}{\sunmonth{मकरः}{6}{}{पौषः\\[-8pt]{\scriptsize (हेमन्त-ऋतुः)}}{इन्दुः}{विकारी}{उत्तरायणम्}{हेमन्त-ऋतुः}}
{\sunmoonsrdata{06:39}{18:00}{03:20(+1)}{14:06}
{\kalas{04:58 05:48 09:40 08:55 10:26 16:29 11:11 13:27 15:44 17:14 18:50 21:09 22:44 01:54(+1)}}}
{\tnykdata{\mbox{\raisebox{-1pt}{\moon[scale=0.8]{26}}\hspace{2pt}कृष्ण-एकादशी\To{}\textsf{48-37 (02:06(+1))}}\hspace{1ex}}%
{\mbox{अनूराधा\To{}\textsf{42-8 (23:30)}}}{\mbox{वृश्चिक-राशिः}}%
{\mbox{गण्डः\To{}\textsf{3-18 (07:58)}}\hspace{1ex}\mbox{वृद्धिः\To{}\textsf{58-55 (06:13(+1))}}\hspace{1ex}\mbox{ध्रुवः\Too{}}}%
{\mbox{बवः\To{}\textsf{19-26 (14:25)}}\hspace{1ex}\mbox{बालवः\To{}\textsf{48-37 (02:06(+1))}}\hspace{1ex}\mbox{कौलवः\Too{}}}{\scriptsize }
}
{\rygdata{08:04--09:29}{10:54--12:19}{13:44--15:09}}
{सर्व-षट्तिला-एकादशी\eventsep मकरायणम्\textsf{}{\RIGHTarrow}\textsf{20:24}}
{Mon} 
\caldata{JANUARY}{21}{\sunmonth{मकरः}{7}{}{पौषः\\[-8pt]{\scriptsize (हेमन्त-ऋतुः)}}{भौमः}{विकारी}{उत्तरायणम्}{हेमन्त-ऋतुः}}
{\sunmoonsrdata{06:39}{18:00}{04:17(+1)}{14:57}
{\kalas{04:58 05:48 09:41 08:55 10:26 16:29 11:11 13:28 15:44 17:15 18:51 21:10 22:45 01:54(+1)}}}
{\tnykdata{\mbox{\raisebox{-1pt}{\moon[scale=0.8]{27}}\hspace{2pt}कृष्ण-द्वादशी\To{}\textsf{47-44 (01:45(+1))}}\hspace{1ex}}%
{\mbox{ज्येष्ठा\To{}\textsf{42-40 (23:43)}}}{\mbox{वृश्चिक-राशिः \RIGHTarrow \textsf{23:43}}}%
{\mbox{ध्रुवः\To{}\textsf{55-20 (04:47(+1))}}\hspace{1ex}\mbox{व्याघातः\Too{}}}%
{\mbox{कौलवः\To{}\textsf{18-3 (13:52)}}\hspace{1ex}\mbox{तैतिलः\To{}\textsf{47-44 (01:45(+1))}}\hspace{1ex}\mbox{गरः\Too{}}}{\scriptsize }
}
{\rygdata{15:10--16:35}{09:29--10:54}{12:20--13:45}}
{हरिवासरः\textsf{}{\RIGHTarrow}\textsf{07:58}\eventsep सेङ्गालिपुरम्~मुत्तण्णावाळ्~आराधना~\#{१२७}}
{Tue} 
\caldata{JANUARY}{22}{\sunmonth{मकरः}{8}{}{पौषः\\[-8pt]{\scriptsize (हेमन्त-ऋतुः)}}{सौम्यः}{विकारी}{उत्तरायणम्}{हेमन्त-ऋतुः}}
{\sunmoonsrdata{06:39}{18:01}{05:12(+1)}{15:49}
{\kalas{04:58 05:48 09:41 08:55 10:26 16:30 11:12 13:28 15:44 17:15 18:51 21:10 22:45 01:55(+1)}}}
{\tnykdata{\mbox{\raisebox{-1pt}{\moon[scale=0.8]{28}}\hspace{2pt}कृष्ण-त्रयोदशी\To{}\textsf{47-53 (01:49(+1))}}\hspace{1ex}}%
{\mbox{मूला\To{}\textsf{44-11 (00:20(+1))}}}{\mbox{धनुर्-राशिः}}%
{\mbox{व्याघातः\To{}\textsf{52-32 (03:40(+1))}}\hspace{1ex}\mbox{हर्षणः\Too{}}}%
{\mbox{गरः\To{}\textsf{17-41 (13:44)}}\hspace{1ex}\mbox{वणिजः\To{}\textsf{47-53 (01:49(+1))}}\hspace{1ex}\mbox{विष्टिः\Too{}}}{\scriptsize }
}
{\rygdata{12:20--13:45}{08:04--09:29}{10:55--12:20}}
{प्रदोष-व्रतम्}
{Wed} 
\caldata{JANUARY}{23}{\sunmonth{मकरः}{9}{}{पौषः\\[-8pt]{\scriptsize (हेमन्त-ऋतुः)}}{गुरुः}{विकारी}{उत्तरायणम्}{हेमन्त-ऋतुः}}
{\sunmoonsrdata{06:39}{18:01}{06:05(+1)}{16:43}
{\kalas{04:58 05:48 09:41 08:55 10:26 16:30 11:12 13:28 15:45 17:16 18:52 21:11 22:45 01:55(+1)}}}
{\tnykdata{\mbox{\raisebox{-1pt}{\moon[scale=0.8]{29}}\hspace{2pt}कृष्ण-चतुर्दशी\To{}\textsf{49-5 (02:17(+1))}}\hspace{1ex}}%
{\mbox{पूर्वाषाढा\To{}\textsf{46-43 (01:21(+1))}}}{\mbox{धनुर्-राशिः}}%
{\mbox{हर्षणः\To{}\textsf{50-33 (02:52(+1))}}\hspace{1ex}\mbox{वज्रम्\Too{}}}%
{\mbox{विष्टिः\To{}\textsf{18-21 (14:00)}}\hspace{1ex}\mbox{शकुनिः\To{}\textsf{49-5 (02:17(+1))}}\hspace{1ex}\mbox{चतुष्पात्\Too{}}}{\scriptsize }
}
{\rygdata{13:45--15:11}{06:39--08:04}{09:30--10:55}}
{मासशिवरात्रिः}
{Thu} 
\caldata{JANUARY}{24}{\sunmonth{मकरः}{10}{}{पौषः\\[-8pt]{\scriptsize (हेमन्त-ऋतुः)}}{भृगुः}{विकारी}{उत्तरायणम्}{हेमन्त-ऋतुः}}
{\sunmoonsrdata{06:39}{18:02}{---}{17:37}
{\kalas{04:58 05:49 09:41 08:56 10:27 16:31 11:12 13:29 15:45 17:16 18:52 21:11 22:46 01:55(+1)}}}
{\tnykdata{\mbox{\raisebox{-1pt}{\moon[scale=0.8]{30}}\hspace{2pt}अमावास्या\To{}\textsf{51-21 (03:11(+1))}}\hspace{1ex}}%
{\mbox{उत्तराषाढा\To{}\textsf{50-16 (02:46(+1))}}}{\mbox{धनुर्-राशिः \RIGHTarrow \textsf{07:40}}}%
{\mbox{वज्रम्\To{}\textsf{49-22 (02:24(+1))}}\hspace{1ex}\mbox{सिद्धिः\Too{}}}%
{\mbox{चतुष्पात्\To{}\textsf{20-5 (14:41)}}\hspace{1ex}\mbox{नाग\To{}\textsf{51-21 (03:11(+1))}}\hspace{1ex}\mbox{किंस्तुघ्नः\Too{}}}{\scriptsize }
}
{\rygdata{10:55--12:20}{15:11--16:36}{08:04--09:30}}
{दर्भ-सङ्ग्रहः\eventsep मौनि (पौष/मकर) अमावस्या\eventsep \tamil{தை~வெள்ளிக்கிழமை}\eventsep \tamil{திருநெல்வேலி நெல்லையப்பர் பத்ர தீப திருவிழா}}
{Fri} 
\caldata{JANUARY}{25}{\sunmonth{मकरः}{11}{}{माघः\\[-8pt]{\scriptsize (शिशिर-ऋतुः)}}{स्थिरः}{विकारी}{उत्तरायणम्}{हेमन्त-ऋतुः}}
{\sunmoonrsdata{06:39}{18:02}{06:55}{18:30}
{\kalas{04:58 05:49 09:41 08:56 10:27 16:31 11:12 13:29 15:46 17:17 18:53 21:11 22:46 01:55(+1)}}}
{\tnykdata{\mbox{\raisebox{-1pt}{\moon[scale=0.8]{1}}\hspace{2pt}शुक्ल-प्रथमा\To{}\textsf{54-40 (04:31(+1))}}\hspace{1ex}}%
{\mbox{श्रवणम्\To{}\textsf{54-51 (04:35(+1))}}}{\mbox{मकर-राशिः}}%
{\mbox{सिद्धिः\To{}\textsf{49-0 (02:15(+1))}}\hspace{1ex}\mbox{व्यतीपातः\Too{}}}%
{\mbox{किंस्तुघ्नः\To{}\textsf{22-52 (15:48)}}\hspace{1ex}\mbox{बवः\To{}\textsf{54-40 (04:31(+1))}}\hspace{1ex}\mbox{बालवः\Too{}}}{\scriptsize }
}
{\rygdata{09:30--10:55}{13:46--15:11}{06:39--08:04}}
{श्यामळानवरात्रि-आरम्भः}
{Sat} 
\caldata{JANUARY}{26}{\sunmonth{मकरः}{12}{}{माघः\\[-8pt]{\scriptsize (शिशिर-ऋतुः)}}{भानुः}{विकारी}{उत्तरायणम्}{हेमन्त-ऋतुः}}
{\sunmoonrsdata{06:39}{18:03}{07:41}{19:22}
{\kalas{04:58 05:49 09:41 08:56 10:27 16:32 11:12 13:29 15:46 17:17 18:53 21:12 22:46 01:55(+1)}}}
{\tnykdata{\mbox{\raisebox{-1pt}{\moon[scale=0.8]{2}}\hspace{2pt}शुक्ल-द्वितीया\To{}\textsf{59-0 (06:15(+1))}}\hspace{1ex}}%
{\mbox{श्रविष्ठा\To{}अहोरात्रम्}}{\mbox{मकर-राशिः \RIGHTarrow \textsf{17:39}}}%
{\mbox{व्यतीपातः\To{}\textsf{49-24 (02:25(+1))}}\hspace{1ex}\mbox{वरीयान्\Too{}}}%
{\mbox{बालवः\To{}\textsf{26-43 (17:20)}}\hspace{1ex}\mbox{कौलवः\To{}\textsf{59-0 (06:15(+1))}}\hspace{1ex}\mbox{तैतिलः\Too{}}}{\scriptsize }
}
{\rygdata{16:37--18:03}{12:21--13:46}{15:12--16:37}}
{चन्द्र-दर्शनम्\eventsep व्यतीपात-श्राद्धम्}
{Sun} 
\caldata{JANUARY}{27}{\sunmonth{मकरः}{13}{}{माघः\\[-8pt]{\scriptsize (शिशिर-ऋतुः)}}{इन्दुः}{विकारी}{उत्तरायणम्}{हेमन्त-ऋतुः}}
{\sunmoonrsdata{06:39}{18:03}{08:24}{20:11}
{\kalas{04:58 05:49 09:41 08:56 10:27 16:32 11:13 13:30 15:46 17:18 18:54 21:12 22:47 01:56(+1)}}}
{\tnykdata{\mbox{\raisebox{-1pt}{\moon[scale=0.8]{3}}\hspace{2pt}शुक्ल-तृतीया\To{}अहोरात्रम् (त्रिदिनस्पृक्)}}%
{\mbox{श्रविष्ठा\To{}\textsf{0-24 (06:49)}}}{\mbox{कुम्भ-राशिः}}%
{\mbox{वरीयान्\To{}\textsf{50-31 (02:51(+1))}}\hspace{1ex}\mbox{परिघः\Too{}}}%
{\mbox{तैतिलः\To{}\textsf{31-32 (19:16)}}\hspace{1ex}\mbox{गरः\Too{}}}{\scriptsize }
}
{\rygdata{08:04--09:30}{10:56--12:21}{13:47--15:12}}
{}
{Mon} 
\caldata{JANUARY}{28}{\sunmonth{मकरः}{14}{}{माघः\\[-8pt]{\scriptsize (शिशिर-ऋतुः)}}{भौमः}{विकारी}{उत्तरायणम्}{हेमन्त-ऋतुः}}
{\sunmoonrsdata{06:39}{18:04}{09:03}{20:58}
{\kalas{04:58 05:49 09:42 08:56 10:27 16:32 11:13 13:30 15:47 17:18 18:54 21:12 22:47 01:56(+1)}}}
{\tnykdata{\mbox{\raisebox{-1pt}{\moon[scale=0.8]{3}}\hspace{2pt}शुक्ल-तृतीया\To{}\textsf{4-17 (08:22)}}\hspace{1ex}}%
{\mbox{शतभिषक्\To{}\textsf{6-49 (09:23)}}}{\mbox{कुम्भ-राशिः \RIGHTarrow \textsf{05:30(+1)}}}%
{\mbox{परिघः\To{}\textsf{52-12 (03:32(+1))}}\hspace{1ex}\mbox{शिवः\Too{}}}%
{\mbox{गरः\To{}\textsf{4-17 (08:22)}}\hspace{1ex}\mbox{वणिजः\To{}\textsf{37-12 (21:32)}}\hspace{1ex}\mbox{विष्टिः\Too{}}}{\scriptsize }
}
{\rygdata{15:12--16:38}{09:30--10:56}{12:21--13:47}}
{\tamil{அப்பூதியடிகள் நாயனார் (24) குருபூஜை}\eventsep वरकुन्द-चतुर्थी}
{Tue} 
\caldata{JANUARY}{29}{\sunmonth{मकरः}{15}{}{माघः\\[-8pt]{\scriptsize (शिशिर-ऋतुः)}}{सौम्यः}{विकारी}{उत्तरायणम्}{हेमन्त-ऋतुः}}
{\sunmoonrsdata{06:39}{18:04}{09:41}{21:44}
{\kalas{04:58 05:48 09:42 08:56 10:27 16:33 11:13 13:30 15:47 17:18 18:54 21:13 22:47 01:56(+1)}}}
{\tnykdata{\mbox{\raisebox{-1pt}{\moon[scale=0.8]{4}}\hspace{2pt}शुक्ल-चतुर्थी\To{}\textsf{10-17 (10:46)}}\hspace{1ex}}%
{\mbox{पूर्वप्रोष्ठपदा\To{}\textsf{13-56 (12:13)}}}{\mbox{मीन-राशिः}}%
{\mbox{शिवः\To{}\textsf{54-16 (04:22(+1))}}\hspace{1ex}\mbox{सिद्धिः\Too{}}}%
{\mbox{विष्टिः\To{}\textsf{10-17 (10:46)}}\hspace{1ex}\mbox{बवः\To{}\textsf{43-27 (00:02(+1))}}\hspace{1ex}\mbox{बालवः\Too{}}}{\scriptsize }
}
{\rygdata{12:21--13:47}{08:04--09:30}{10:56--12:21}}
{मार्कण्डेय~जयन्ती}
{Wed} 
\caldata{JANUARY}{30}{\sunmonth{मकरः}{16}{}{माघः\\[-8pt]{\scriptsize (शिशिर-ऋतुः)}}{गुरुः}{विकारी}{उत्तरायणम्}{हेमन्त-ऋतुः}}
{\sunmoonrsdata{06:39}{18:05}{10:17}{22:29}
{\kalas{04:58 05:48 09:42 08:56 10:27 16:33 11:13 13:30 15:47 17:19 18:55 21:13 22:47 01:56(+1)}}}
{\tnykdata{\mbox{\raisebox{-1pt}{\moon[scale=0.8]{5}}\hspace{2pt}शुक्ल-पञ्चमी\To{}\textsf{16-41 (13:19)}}\hspace{1ex}}%
{\mbox{उत्तरप्रोष्ठपदा\To{}\textsf{21-24 (15:12)}}}{\mbox{मीन-राशिः}}%
{\mbox{सिद्धिः\To{}\textsf{56-26 (05:13(+1))}}\hspace{1ex}\mbox{साध्यः\Too{}}}%
{\mbox{बालवः\To{}\textsf{16-41 (13:19)}}\hspace{1ex}\mbox{कौलवः\To{}\textsf{49-53 (02:36(+1))}}\hspace{1ex}\mbox{तैतिलः\Too{}}}{\scriptsize }
}
{\rygdata{13:47--15:13}{06:39--08:04}{09:30--10:56}}
{षष्ठी-व्रतम्\eventsep माघी~सरस्वती-पूजा\eventsep सर्प-पूजा\eventsep वसन्त-श्री-पञ्चमी\eventsep श्रीराम-वनवास-गमनम्}
{Thu} 
\caldata{JANUARY}{31}{\sunmonth{मकरः}{17}{}{माघः\\[-8pt]{\scriptsize (शिशिर-ऋतुः)}}{भृगुः}{विकारी}{उत्तरायणम्}{हेमन्त-ऋतुः}}
{\sunmoonrsdata{06:39}{18:05}{10:53}{23:14}
{\kalas{04:58 05:48 09:42 08:56 10:27 16:34 11:13 13:30 15:48 17:19 18:55 21:13 22:48 01:56(+1)}}}
{\tnykdata{\mbox{\raisebox{-1pt}{\moon[scale=0.8]{6}}\hspace{2pt}शुक्ल-षष्ठी\To{}\textsf{23-2 (15:52)}}\hspace{1ex}}%
{\mbox{रेवती\To{}\textsf{28-47 (18:10)}}}{\mbox{मीन-राशिः \RIGHTarrow \textsf{18:10}}}%
{\mbox{साध्यः\To{}\textsf{58-21 (05:59(+1))}}\hspace{1ex}\mbox{शुभः\Too{}}}%
{\mbox{तैतिलः\To{}\textsf{23-2 (15:52)}}\hspace{1ex}\mbox{गरः\To{}\textsf{56-2 (05:04(+1))}}\hspace{1ex}\mbox{वणिजः\Too{}}}{\scriptsize }
}
{\rygdata{10:56--12:22}{15:13--16:39}{08:04--09:30}}
{भृगुरेवती-पुण्यकालः\eventsep \tamil{கலிக்கம்ப நாயனார் (42) குருபூஜை}\eventsep \tamil{தை~வெள்ளிக்கிழமை}}
{Fri} 
\end{document}
